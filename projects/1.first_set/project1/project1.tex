\documentclass{article}


%Packages-------------------------------------
\usepackage{framed}
\usepackage{graphicx}
\usepackage{hyperref}
\usepackage{amsmath, amsfonts, amssymb}
\usepackage[inline]{enumitem}
\usepackage[localise=on]{xepersian}
\settextfont[Scale=1]{BNazanin}



\begin{document}
	\title{%
		\includegraphics[width=0.3\textwidth]{pics/python_civiliup.jpg}~ 
		\\
		پروژه اول دوره پایتون
	}
	\author{کمال فراهانی}
	\maketitle
	
	\section*{پروژه‌ی اول: دفترچه تلفن}
	در این پروژه برنامه مدیریت دفترچه تلفن را با زبان پایتون خواهید نوشت. برنامه باید دارای قابلیت‌های زیر باشد:
	\begin{enumerate}
		\item 
			در هنگام اجرای برنامه، برنامه پیام منوی اصلی را به کاربر نشان می‌دهد، که به کاربر اعلام می‌کند که می‌تواند از بین پنج گزینه
		\begin{enumerate*}
			\item اضافه کردن شماره به دفترچه
			\item جستجوی شماره فرد با نام
			\item جستجوی نام فرد با شماره تلفن
			\item حذف شماره فرد
			\item خروج از برنامه
		\end{enumerate*}
	یکی را انتخاب کند.
		\item	
	در صورت انتخاب گزینه‌ای غیر از خروج، برنامه باید خواسته کاربر را انجام داده و پس از اتمام کار دوباره به منوی اصلی برگردد و پیام منوی اصلی را به کاربر نشان دهد.
	
	\item 
	برنامه در هنگام اضافه کردن شماره جدید برای کاربر جدید باید بررسی کند که کاراکترهای شماره وارد شده همگی عدد باشند و طول رشته شماره بین ۱۰ تا ۲۰ کاراکتر باشد و در صورتی که غیر از این بود، به کاربر اعلام کند که قالب ورودی شماره نامعبر است و باید شماره را دوباره وارد کند.
	
	\item 
	باید مکانیزمی وجود داشته باشد که در صورتی که کاربر در هنگام اجرای هر کدام از چهار عملیات‌ منصرف شد، بتواند به منوی اصلی برگردد.
	\end{enumerate}

	در این پروژه اطلاعات وارد شده در برنامه تا پایان اجرای برنامه نگهداری خواهند شد اما بعد از خروج از برنامه اطلاعات از دست می‌رود. این مشکل در پروژه‌های آینده و پس از یادگیری بخش نوشتن در فایل‌های متنی رفع خواهد شد.
\end{document}