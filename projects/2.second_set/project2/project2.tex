\documentclass{article}


%Packages-------------------------------------
\usepackage{framed}
\usepackage{graphicx}
\usepackage{hyperref}
\usepackage{amsmath, amsfonts, amssymb}
\usepackage[inline]{enumitem}
\usepackage[localise=on]{xepersian}
\settextfont[Scale=1]{BNazanin}



\begin{document}
	\title{%
		\includegraphics[width=0.3\textwidth]{pics/python_civiliup.jpg}~ 
		\\
		پروژه دوم دوره پایتون
	}
	\author{کمال فراهانی}
	\maketitle
	
	\section*{پروژه دوم: دفترچه تلفن شی‌گرا با قابلیت ذخیره}
	در پروژه قبلی برنامه مدیریت دفترچه تلفن را به زبان پایتون نوشتیم، در این پروژه قصد بهبود و رفع مشکلات پروژه قبلی را داریم. ابتدا باید پروژه قبلی را به کمک کلاس‌ها به صورت شی‌گرا طراحی کنید. کلاسی به نام
	\lr{Contact}
	تعریف کنید و از آن برای ساخت اشیا نمایش دهنده افراد در برنامه استفاده کنید، به این موضوع فکر کنید که کلاس 
	\lr{Contact}
	باید چه ویژگی‌ها و رفتارهایی داشته باشد، و آنها را پیاده سازی بکنید.
	سپس به کمک فایل‌ها هر بار که مخاطب جدید به برنامه اضافه می شود اطلاعات آن فرد را با فرمت دلخواه به یک فایل اضافه کنید و هربار که مخاطبی حذف می‌شود اطلاعات آن را از فایل حذف کنید. در هنگام شروع برنامه این اطلاعات را بازیابی کنید، به این ترتیب مشکل ذخیره اطلاعات که در برنامه قبلی وجود داشت حل می شود.
\end{document}